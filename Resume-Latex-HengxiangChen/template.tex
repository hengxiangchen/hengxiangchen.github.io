%%%%%%%%%%%%%%%%%%%%%%%%%%%%%%%%%%%%%%%%%
% Medium Length Professional CV
% LaTeX Template
% Version 3.0 (December 17, 2022)
%
% This template originates from:
% https://www.LaTeXTemplates.com
%
% Author:
% Vel (vel@latextemplates.com)
%
% Original author:
% Trey Hunner (http://www.treyhunner.com/)
%
% License:
% CC BY-NC-SA 4.0 (https://creativecommons.org/licenses/by-nc-sa/4.0/)
%
%%%%%%%%%%%%%%%%%%%%%%%%%%%%%%%%%%%%%%%%%

%----------------------------------------------------------------------------------------
%	PACKAGES AND OTHER DOCUMENT CONFIGURATIONS
%----------------------------------------------------------------------------------------

\documentclass[
	%a4paper, % Uncomment for A4 paper size (default is US letter)
	11pt, % Default font size, can use 10pt, 11pt or 12pt
]{resume} % Use the resume class

\usepackage{ebgaramond} % Use the EB Garamond font

\usepackage[colorlinks=true, urlcolor=blue, linkcolor=black]{hyperref}

%------------------------------------------------

\name{Hengxiang Chen} % Your name to appear at the top

% You can use the \address command up to 3 times for 3 different addresses or pieces of contact information
% Any new lines (\\) you use in the \address commands will be converted to symbols, so each address will appear as a single line.

%\address{Shenzhen, China} % Main address

\address{School of Artificial Intelligence, SZTU \\ Shenzhen, China} % A secondary address (optional)

\address{(86)~$\cdot$~15816659727\\ {\href{mailto:hengxiangchen428@gmail.com}{hengxiangchen428@gmail.com}}} % Contact information


\address{\href{https://hency-727.github.io/}{hency-727.github.io}}

%----------------------------------------------------------------------------------------

\begin{document}

%----------------------------------------------------------------------------------------
%	EDUCATION SECTION
%----------------------------------------------------------------------------------------

\begin{rSection}{Education}
	
	\textbf{Hong Kong University of Science and Technology (Guangzhou)} \hfill September 2026 (Expected)
	\newline M.Phil. / Ph.D. (Full Scholarship) in Robotics and Autonomous Systems (Offer Accepted)

	
	\textbf{Shenzhen Technology University}\hfill \textit{Shenzhen, China}  \\ 
	{B.S. in Vehicle Engineering}\hfill {September 2021 - June 2025} \\
	Honor of Headmaster's Scholarship and Best Ten Graduated Student candidates \\
	Member of X-Talent Program(Academic Training Program of SZTU)  \\
	Overall GPA: 3.55/4.5 with 10/112
	
	\textbf{Hochschule Coburg}\hfill \textit{Kronach, Germany}  \\ 
	{Exchange intern of Autonomous Driving (Master-Level)}\hfill {September 2021 - June 2025}

\end{rSection}

%----------------------------------------------------------------------------------------
%	WORK EXPERIENCE SECTION
%----------------------------------------------------------------------------------------

\begin{rSection}{Publications}{*Equal Contribution}{}{}
	\item Z.~Guo\textsuperscript{*}, \textbf{H.~Chen}\textsuperscript{*}, Q.~Li, \textit{et al.}, ``Octopi-X: Cross\mbox{-}Modal Robotic Perception with a Large Vision--Language Model for Physical Property Inference,'' in \textit{IROS 2025 workshop}. (Oral\&Poster Presentation) \href{https://openreview.net/group?id=IEEE.org/IROS/2025/Workshop/Tactile_Sensing/Authors&referrer=%5BHomepage%5D(%2F)}{\textit{[Openreview Paper]}} 
	
	\item Z.~Guo\textsuperscript{*}, \textbf{H.~Chen}\textsuperscript{*}, Q.~Li, \textit{et al.}, ``Cross\mbox{-}Modal Robotic Perception with a Large Vision--Language Model for Physical Property Inference,'' in \textit{CLAW 2025}. (Accepted) \href{https://arxiv.org/pdf/2506.19303}{\textit{[arXiv Paper:2506.19303]}}
	\item Z.~Feng, \textbf{H.~Chen}, L.~Chen, X.~Mou, ``Path Planning Algorithm Comparison Analysis for Wireless AUVs Energy\mbox{-}Sharing System,'' in \textit{IEEE Industrial Electronic Technology News (ITeN)}, 2023. (Accepted) \href{https://ieeexplore.ieee.org/document/10311674}{\textit{[IEEE Paper]}} 

\end{rSection}


\begin{rSection}{Experience}

	\begin{rSubsection}{Arbeit Gruppe Dexterous Robotics Lab, SZTU}{September 2024 - Present}{Research Assistant under \href{https://tusz-agdr.github.io/author/qiang-li/ and Phd}{Prof. Qiang Li} and  \href{https://www.linkedin.com/in/nutan-chen-17678678/?originalSubdomain=de}{Dr. Nutan Chen}}{Shenzhen, China}
		\item Research on Robot Learning.
	\end{rSubsection}

%------------------------------------------------

	\begin{rSubsection}{VALEO}{March 2024 - August 2024}{\textit{R\&D Trainee under the supervision of \href{https://www.linkedin.com/in/yongwei-bryan-yang-580392a5/?originalSubdomain=de}{System Engineer Yongwei Yang}}}{Kronach, Germany}
		\item Quantitatively analyzes the impact of latency and vehicle speed on remote urban driving control using statistical methods based on simulation and real-world vehicle data.
	\end{rSubsection}
	
%------------------------------------------------

	\begin{rSubsection}{Intelligent Automotive Research Team, SZTU}{March 2024 - Aug 2024}{Undergraduate Student under \href{https://utl.sztu.edu.cn/info/1286/1978.htm}{Prof. Heyan Li} and \href{https://utl.sztu.edu.cn/info/1286/1995.htm}{Prof. Xiaolin Mou}}{Shenzhen, China}
		\item Research on Vehicle Control and Path Planning. 
		\item Team Technology Leader of AutoBots(Smart Racing Car Team).
	\end{rSubsection}

\end{rSection}


\begin{rSection}{COMPETITIONS}
		
	\begin{rSubsection}
		{\href{https://github.com/Hency-727/racecar_outdoor_ros_competition2022}{Chinese Robotics and Artificial Intelligence Competition (Intelligent Driving)}}
		{Hainan, China}
		{Team Leader, \textbf{5th Place (National First Prize)}}
		{June 2023}
		\item Participated in the development of ROS-based autonomous racing system, responsible for perception and planning modules.
	\end{rSubsection}

%------------------------------------------------
	\begin{rSubsection}
		{\href{https://github.com/Hency-727/racecar_outdoor_ros_competition2022}{Chinese Outdoor ROS Autonomous Racing Competition}}
		{Shenzhen, China}
		{Team Leader, \textbf{3th Place (National First Prize)}}
		{December 2022}
		\item Developed intelligent driving algorithms for multi-sensor fusion and real-time decision-making. 
	\end{rSubsection}

\end{rSection}


%----------------------------------------------------------------------------------------
%	TECHNICAL STRENGTHS SECTION
%----------------------------------------------------------------------------------------


\begin{rSection}{Technical Strengths}
	\begin{tabular}{@{} >{\bfseries}l @{\hspace{6ex}} l @{}}
		Programming Languages & Python, C/C++, MATLAB, Bash \\
		Frameworks \& Libraries & ROS/ROS2, PyTorch, OpenCV\\
		Tools \& Platforms & Linux (Ubuntu), Git, Docker, Conda, VSCode, Gazebo\\
		Robotics \& Sensors & Kinova Gen3, RealSense D435i/D455i, GelSight Mini
	\end{tabular}
\end{rSection}

%----------------------------------------------------------------------------------------
%	EXAMPLE SECTION
%----------------------------------------------------------------------------------------

%\begin{rSection}{Section Name}

	%Section content\ldots

%\end{rSection}

%----------------------------------------------------------------------------------------

\end{document}
